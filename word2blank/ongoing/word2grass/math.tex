\documentclass[a4paper,11pt]{article}
\usepackage{color,xcolor,ucs}
\usepackage[top=1.2in, bottom=1.2in, left = 1in, right = 1in]{geometry}
\usepackage[linkcolor=black,colorlinks=true,urlcolor=blue]{hyperref}

\usepackage{amsmath}
\title{Analogy using vector transport}
\author{Word2grass}

\begin{document}
\maketitle
\begin{center}
\rule{\textwidth}{1pt}
\end{center}

\noindent 
Analogy task looks like this - $a:b :: c:d$
where $a,b,c$ are three words represented by points on the grassmanian manifold. 
Now, let me denote $a,b,c$ with orthonormal matrices $A,B,C$ respectively.

\noindent Steps to find the unknown subspace $D$ representing the word $d$ - 

\begin{itemize}
  \item Use the closed form equation of the geodesic connecting $a$ and $b$ to get the tangent vector at point $A$. The closed form equation is given by 
  \[\gamma(t) = XV\cos(\Theta t) + U\sin(\theta t)\] where $\gamma(0)=A$ , $\gamma(1)=B$ and $\Theta = \tanh{\Sigma}$. The quantities $U,\Sigma, V$ are gotten after the SVD decomposition of projection of $B(A^TB)^{-1}$ onto the orthogonal complement of X, i.e.,
  \[U\Sigma V^T = (I - AA^T)B(A^TB)^{-1}\]
  The tangent vector that we need is basically $\dot \gamma(0)$. Let us give it a better name $T_A$. So $T_A$ can be calculated simply using $U \Theta V^T$. Infact, $\Theta$ is the diagonal matrix having all the principal angles between subspace $A$ and $B$. 
  \item We also calculate the geodesic length of the curve connecting the two subspaces. This turns out to be $\sqrt{\sum_i \theta_i^{2}}$ where $\theta_i$ are the principal angles between the two subspaces. Let us call it $L$.
  \item Now that we have $T_A$, we use the vector transport equation to parallely translate it along the geodesic connecting $A$ and $C$. The closed form of the equation is given by 
  \[T_A(t) = (-AV \sin(\Sigma t)U^T + U\cos (\Sigma t)U^T + (I - UU^T))T_A \] where $T_A(t)$ represents the tangent vector after being transported to a point $\alpha(t)$ where $alpha$ represents the geodesic. $U,\Sigma, V$ are gotten after SVD decomposition of $T_A$, i.e, $U \Sigma V^T = T_A$. We require $\tau T_A(1)$ and let us call it $T_C$. 
  \item Next step is to use this transported vector at subspace $C$ to move along the geodesic till we have covered a distance of length $L$. The closed form of this equation is given by - 
  \[\beta(t) = XV\cos(\Theta t) + U\sin(\theta t) \] where $\beta(0) = C$ and $U,\Theta, V$ are gotten after SVD decomposition of the parallel transported tangent vector $T_C$, i.e, $U\Theta V^T = T_{C}$.
\end{itemize}

\noindent So, the point we get after traversing the curve $\beta$ till length $L$ at a velocity equal to $T_{C}$ is our required subspace $D$. Now we search for all words that are close to $D$.

\end{document}