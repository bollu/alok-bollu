% File tacl2018.tex
% Aug 3, 2018

% The English content of this file was modified from various *ACL instructions
% by Lillian Lee and Kristina Toutanova
%
% LaTeXery is all adapted from acl2018.sty.

\documentclass[11pt,a4paper]{article}
\usepackage[hyperref]{tacl2018} % use ``nohyperref'' to disable  hyperref
\usepackage{times,latexsym}
\usepackage{amsmath}
\usepackage{amssymb}
\usepackage{url}
\usepackage[T1]{fontenc}

\taclfinalfalse % For camera-ready, replace "\taclfinalfalse" with
% "\taclfinalcopy"

%%%%
%%%% Material in this block can be removed by TACL authors.
% It consists of things specific to generating TACL instructions
\usepackage{xspace,mfirstuc,tabulary}

\newcommand{\ex}[1]{{\sf #1}}
\newcommand{\wordtovec}{\texttt{Word2Vec} }
\newcommand{\wordtoman}{\texttt{Word2Man} }
\newcommand{\realn}{$\mathbb{R}^n$}

%
\iftaclfinal
\newcommand{\taclpaper}{camera-ready\xspace}
\newcommand{\taclpapers}{camera-readies\xspace}
\newcommand{\Taclpaper}{Camera-ready\xspace}
\newcommand{\Taclpapers}{Camera-readies\xspace}
\else
\newcommand{\taclpaper}{submission\xspace}
\newcommand{\taclpapers}{{\taclpaper}s\xspace}
\newcommand{\Taclpaper}{Submission\xspace}
\newcommand{\Taclpapers}{{\Taclpaper}s\xspace}
\fi

\newif\iftaclinstructions
\taclinstructionsfalse
\iftaclinstructions
\renewcommand{\confidential}{}
\renewcommand{\anonsubtext}{(No author info supplied here, for consistency with
TACL-submission anonymization requirements)}
\fi
%%%% End TACL-instructions-specific macro block
%%%%

\title{$Word2Man$: Length does matter}


% The command \taclfinalfalse suppresses display of the contents of the
% \author{...} command in the generated pdf.
% Replacing that command with "\taclfinalcopy" reveals the author info in the
% generated pdf.
% See tacl2018.sty for other ways to set author info.
\author{Anonymous authors}

\date{}

\begin{document}
\maketitle


\begin{abstract}
Word embeddings have been studied in great detail by the NLP community,
starting with ??. The first vector-based embeddings began with Word2Vec(\cite{}),
which has now become the de-facto style of word embeddings.

However, the name is a misnomer --- The properties used in the training
of \wordtovec depend on none of the vector space structure of \realn.
Rather, \wordtovec uses the distance and dot-product structure available
in \realn. Luckily, mathematicians are hard working and have already
generalized these notions for us --- Into those of Reimanninan Manifolds
and Metric spaces, which capture notions of dot product and
distances respectively. We use this generalization to study word
embeddings that have been proposed over the years, including 
Poincare Embeddings and ELMo, two state-of-the-art word embeddings.

We first analyze existing word embeddings to gain a geometric
viewpoint of current word embeddings. We use tools from topological
data anlysis to post-facto understand the geometric structure 
of word embeddings.

We use the classification of surfaces and low dimensional topologies
that have been created by mathematicians to analyze many canditate spaces
for word embeddings. We compare these spaces, and reach the conclusion
that for general word embeddings, ??? is the best choice.

We also provide explanations for why ??? provides the best embeddings,
in terms of geometric properties which govern the manifold.

they embedded 
  
\end{abstract}


\section{Introduction}


Gradient descent over reimannian manifolds have been explored in
\cite{bonnabel2013stochastic}, but it seems that embedding into non euclidian
spaces have been explored to some degree in \cite{nickel2017poincare} where ...


Continuous Bag of Words \cite{mikolov2013efficient} and Skip Gram \cite{mikolov2013distributed}, which constituted the Word2Vec model of word embeddings into vector spaces, became the hallmark of word embeddings with the application of the Vector Space Model \cite{salton1975vector} to creating word vectors. This vector space model for word embeddings became the cornerstone for multiple other developments, including universalization of word vectors in GloVe \cite{pennington2014glove}, modeling context probabilities in the vector space model \cite{peters2018deep}, in using sub-word information for enriching word vectors \cite{bojanowski2017enriching} and in multiple applications due to the ease of computation in a vector space and the relatively high accuracies as compared to other contemporary methods.
% cite other applications here

Given the multitude of applications of the word embedings in a vector space, and the ease of the representation of the distributional hypothesis \cite{firth1975modes}, the nature of the underlying algebra has been essentially ignored, hence neither the efficiency nor the "correctness" of using a vector space has been explored to much detail. This paper attempts to question the approproateness of a vector space as the algebra for word embeddings given the properties often cited (for example "king" - "man" + "woman"; in tandem with the vector addition operation), but the absence of the operation of scalar multiplication.
% cite properties of vector space

The implications of using a vector space without using the scalar multiplication operator eliminates the concept of distance. %how?
Latent heirarchihcal propoerties are not accounted for in Eucledian vector spaces \cite{nickel2017poincare}, and therefore, there is emperical evidence of the use of non linear geometric spaces to exploit the patterns in datasets for higher accuracies. This paper is an attempt to report the performance of two specific generalizations on the vector embeddings model, those being affine spaces and a general algebraic manifold. %please expand on this Siddharth :( 

\section{Word Embeddings}

\Taclpapers that do not comply with this document's instructions
risk
\iftaclfinal
publication delays until the camera-ready is brought into compliance.
\else
rejection without review.
\fi


Submissions should consist of a Portable Document Format (PDF) file formatted
for  \textbf{A4 paper}.\footnote{Prior to the September 2018 submission round, a
different paper size was used.} All necessary fonts should be
included in the  file.

If you promised to provide code or data at submission, specific instructions for
how to access such resources must be provided.  (Typically, a URL to a stable,
resource-specific site suffices.)

All URLs should be manually checked to verify that they
lead to a valid webpage, and to the site that was intended.



\section{\LaTeX\ files}

\LaTeX\ files compliant with these instructions are available at the
Author Guidelines section of the
TACL website, \href{https://www.transacl.org/}
{https://www.transacl.org}.\footnote{Last accessed August 3, 2018.} Use of the
TACL \LaTeX\ files is highly recommended: \emph{MIT Press requires authors to
supply \LaTeX\ source files as part of the publication process}; and
use of the recommended \LaTeX\ files makes conversion to the
required camera-ready format simple.
\iftaclfinal
Specifically, the conversion can be accomplished by as little as: (1) change the
line
\verb+\taclfinalfalse+ to \verb+taclfinalcopy+; (2) add author information;
(3) add acknowledgments.
\fi


The provided files use the hyperref package.  Should you wish to disable it,
change the line
\verb+\usepackage[hyperref]{tacl2018}+
to
\verb+\usepackage[nohyperref]{tacl2018}+.
We mention this because citations or URLs that cross pages can trigger the
compilation error ``{\textbackslash}pdfendlink ended up in different nesting
level than {\textbackslash}pdfstartlink''.  In such cases, you may temporarily
disable the hyperref package and then compile to locate the offending portion of
the tex file; edit to avoid a pagebreak within a link; and then re-enable the
hyperref package.


\section{Length limits}
\label{sec:length}

\iftaclfinal
Camera-ready documents may consist of as many pages of content as allowed by
the Action Editor in their final acceptance letter.
\else
Submissions may consist of seven to ten (7-10) A4 format (not letter) pages of
content.
\fi

The page limit \emph{includes} any appendices. However, references
\iftaclfinal
and acknowledgments
\fi
do not count
toward the page limit.

\iftaclfinal
\else
Exception: Revisions of (b) or (c) submissions may have been allowed
additional pages of content by the prior Action Editor, as specified in their
decision letter.
\fi

\section{Fonts and text size}

Adobe's {Times Roman} font should be used. In \LaTeX2e{} this is accomplished by
putting \verb+\usepackage{times,latexsym}+ in the preamble.\footnote{Should
Times Roman be unavailable to you, use
{Computer Modern Roman} (\LaTeX2e{}'s default).  Note that the latter is about
10\% less dense than Adobe's Times Roman font.}

Font size requirements are listed in Table \ref{tab:font-table}. In addition to
those requirements, the content of figures, tables, equations, etc. must be
of reasonable size and readability.
\begin{table}[t]
\begin{center}
\begin{tabular}{|l|rl|}
\hline \bf Type of Text & \bf Size & \bf Style \\ \hline
paper title & 15 pt & bold \\
\iftaclfinal
author names & 12 pt & bold \\
author affiliation & 12 pt & \\
\else
\fi
the word ``Abstract'' as header & 12 pt & bold \\
abstract text & 10 pt & \\
section titles & 12 pt & bold \\
document text & 11 pt  &\\
captions & 10 pt & \\
%bibliography & 10 pt & \\
footnotes & 9 pt & \\
\hline
\end{tabular}
\end{center}
\caption{\label{tab:font-table} Font requirements}
\end{table}




\section{Page Layout}
\label{ssec:layout}


The margin dimensions for a page in A4 format (21 cm $\times$ 29.7 cm) are given
in Table \ref{tab:margin-table}.  Start the content of all pages directly under
the top margin.
\iftaclfinal
\else
(The confidentiality header (\S\ref{sec:ruler-and-header}) for submissions is an
exception.)
\fi


\begin{table}[ht]
\begin{center}
\begin{tabular}{|l|}  \hline
Left and right margins: 2.5 cm \\
Top margin: 2.5 cm \\
Bottom margin: 2.5 cm \\
Column width: 7.7 cm \\
Column height: 24.7 cm \\
Gap between columns: 0.6 cm \\ \hline
\end{tabular}
\end{center}
\caption{\label{tab:margin-table} Margin requirements}
\end{table}


Papers must be in two-column format and single-spaced.
Allowed exceptions to the two-column format are the title, which must be
centered at the top of the first page;
\iftaclfinal
the author block containing author names and affiliations and addresses, which
must be centered on the top of the first page and placed after the title;
\else
the  confidentiality header (see \S\ref{sec:ruler-and-header}) on submissions;
\fi
and any full-width figures or tables.

Should the pages be numbered?  Yes, for submissions (to facilitate review); but
no, for camera-readies (page numbers will be added at publication time).


{Indent} by about 0.4cm when starting a new paragraph that is not the first in a
section or subsection.

\subsection{The confidentiality header and line-number ruler}
\label{sec:ruler-and-header}
\iftaclfinal
Camera-readies should not include the left- and right-margin line-number rulers
or headers from the submission version.
\else
Each page of the submission should have the header ``\confidentialtext''
centered across both columns in the top margin.

Submissions must include line numbers in the left and right
margins, as demonstrated in the TACL submission-formatting
instructions pdf file, because the line numbering allows reviewers to be very
specific in their comments.\footnote{Authors using Word to prepare their
submissions can create the marginal line numbers by inserting text
boxes containing the line numbers.}
Note that the numbers on the ruler need not line up exactly with the text lines
of the paper. (Indeed, the line numbers generated by the recommended \LaTeX\
files typically do not correspond exactly to the text lines.)
\fi

The presence or absence of the ruler or header should not change the appearance
of any other content on the page.



\begin{table*}[t]
\centering
\begin{tabular}{p{7.8cm}@{\hskip .5cm}p{7.8cm}}
\multicolumn{1}{c}{{\bf Incorrect}} & \multicolumn{1}{c}{{\bf Correct}} \\  \hline
``\ex{(Cardie, 1992) employed learning.}'' &
``\ex{Cardie (1992) employed learning.}'' \\
{The problem}:  ``employed learning.'' is not a sentence.  & Create by
\verb+\citet{+\ldots\verb+}+  or \verb+\newcite{+\ldots\verb+}+. \\
\\  \hline
``\ex{The method of (Cardie, 1992) works.}'' &
``\ex{The method of Cardie (1992) works.}''  \\
{The problem}:  ``The method of was used.'' is not a sentence.  & Create as
above.\\ \\\hline
``\ex{Use the method of (Cardie, 1992).}'' &
``\ex{Use the method of Cardie (1992).}''  \\
{The problem}:  ``Use the method of.'' is not a sentence.  & Create as
above.\\ \\\hline
\ex{Related work exists Lee (1997).} & \ex{Related work exists (Lee,
1997).} \\
{The problem}:  ``Related work exists Lee.'' is not a sentence (unless one
is scolding a Lee). & Create by
\verb+\citep{+\ldots\verb+}+  or \verb+\cite{+\ldots\verb+}+. \\
\\  \hline
\end{tabular}
\caption{\label{tab:cite-commands} Examples of incorrect and correct citation
  format.  Also depicted are citation commands supported by the
  tacl2018.sty file, which is based on the natbib package and
  supports all natbib citation commands.
  The tacl2018.sty file also supports commands defined in previous ACL style
  files
  for compatibility.
  }
\end{table*}





\section{The First Page}
\label{ssec:first}

Center the title, which should be placed 2.5cm from the top of the page,
\iftaclfinal
and author names and affiliations
\fi
across both columns of the first page. Long titles should be typed on two lines
without a blank line intervening.
\iftaclfinal
After the title, include a blank line before the author block.
Do not use only initials for given names, although middle initials are allowed.
Do not put surnames in all capitals.\footnote{Correct: ``Lillian Lee'';
incorrect: Lillian LEE.} Affiliations should include authors' email
addresses. Do not use footnotes for affiliations.
\else
Do not include the paper ID number assigned during the submission process.
\fi

\iftaclfinal
\else
Although submissions should not include any author information, maintain space
for names and affiliations/addresses so that they will fit in the final
(camera-ready)
version.
\fi


Start the abstract at the beginning of the first
column, about 8 cm from the top of the page, with the centered header
``Abstract'' as specified in Table \ref{tab:font-table}.
The width of the abstract text
should be narrower than the width of the columns for the text in the body of the
paper by about 0.6cm on each side.

\section{Section headings}

Use numbered section headings (Arabic numerals) in order to facilitate cross
references. Number subsections with the section number and the subsection number
separated by a dot.



\section{Figures and Tables}

Place figures and tables in the paper near where they are first discussed.

Provide a caption for every illustration. Number each one
sequentially in the form:  ``Figure 1: Caption of the Figure.'' or ``Table 1:
Caption of the Table.''

Authors should ensure that tables and figures do not rely solely on color to
convey critical distinctions and are, in general,  accessible to the
color-blind.



\section{Citations and references}
\label{sec:cite}


\subsection{In-text citations}
\label{sec:in-text-cite}
Use correctly parenthesized author-date citations
(not numbers) in the text. To understand correct parenthesization, obey the
principle that \emph{a sentence containing parenthetical items should remain
grammatical when the parenthesized material is omitted.} Consult Table
\ref{tab:cite-commands} for usage examples.


\iftaclfinal
\else
\subsection{Self-citations}
\label{sec:self-cite}

Citing one's own relevant prior work should be done,  but use the third
person instead of the first person, to preserve anonymity:
\begin{tabular}{l}
Correct: \ex{Zhang (2000) showed ...} \\
Correct: \ex{It has been shown (Zhang, 2000)...} \\
Incorrect: \ex{We (Zhang, 2000) showed ...} \\
Incorrect: \ex{We (Anonymous, 2000) showed ...}
\end{tabular}
\fi

\subsection{References}
\label{sec:references}
Gather the full set of references together under
the boldface heading ``References''. Arrange the references alphabetically
by first author's last/family name, rather than by order of occurrence in the
text.

References to peer-reviewed publications should be given in addition to or
instead of preprint versions. When giving a reference to a preprint, including
arXiv preprints, include the number.

List all authors of a given reference, even if there are dozens; do not
truncate the author list with an ``et al.''  Use full first/given names for
authors, not initials.  Include periods after middle initials.

Titles should have correct capitalization.  For example, change change
``lstm'' or ``Lstm'' to ``LSTM''.\footnote{If using BibTex, apply curly braces
within the title field to preserve intended capitalization.}   Capitalize the
first letter of the first word after a colon or similar punctuation mark.  For
book titles, capitalize the first letter of all main words.  See the
reference entry for \citet{Jurafsky+Martin:2009a} for an example.


We strongly encourage the following, but do not absolutely mandate them:
\begin{itemize}
\item Include DOIs.\footnote{The supplied \LaTeX\ files will
automatically add hyperlinks to the DOI when BibTeX or
BibLateX are invoked if the hyperref package is used and
the doi field is employed in the corresponding bib entries.
The DOI itself will not be separately printed out in that case.}
\item Include the version number when citing arXiv preprints, even if only one
version exists at the time of writing.
For example,\footnote{Bibtex entries for \citet{DBLP:journals/corr/cs-CL-0108005} and
\citet{DBLP:journals/corr/cs-CL-9905001} corresponding to the depicted output
can be found in the supplied sample file {\tt tacl.bib}.  We also cite
the peer-reviewed versions \cite{GOODMAN2001403,P99-1010}, as required.}
note the ``v1'' in the following.
\begin{quote}
Joshua Goodman.  2001.  A bit of progress in language modeling. {\it CoRR},
cs.CL/0108005v1.
\end{quote}
An alternative format is:
\begin{quote}
Rebecca Hwa. 1999. Supervised grammar induction using training data with limited constituent
information. {cs.CL/9905001}. Version 1.
\end{quote}
\end{itemize}

\section{Appendices} Appendices, if any, directly follow the text and the
references.  Recall from Section \ref{sec:length} that {\em appendices count
towards the page
limit.}


\iftaclfinal

\section{Including acknowledgments}
Acknowledgments appear immediately before the references.  Do not number this
section.\footnote{In \LaTeX, one can use {\tt {\textbackslash}section*} instead
of {\tt {\textbackslash}section}.} If you found the reviewers' or Action
Editor's comments helpful, consider acknowledging them.
\else
\fi

\section{Contributors to this document}
\label{sec:contributors}

This document was adapted by Lillian Lee and Kristina Toutanova
from the instructions and files for ACL 2018, by Shay Cohen, Kevin Gimpel, and
Wei Lu. Those files were drawn from earlier *ACL proceedings, including those
for ACL 2017 by Dan Gildea and Min-Yen Kan, NAACL 2017 by Margaret Mitchell,
ACL 2012 by Maggie Li and Michael White, those from ACL 2010 by Jing-Shing
Chang and Philipp Koehn, those for ACL 2008 by Johanna D. Moore, Simone
Teufel, James Allan, and Sadaoki Furui, those for ACL 2005 by Hwee Tou Ng and
Kemal Oflazer, those for ACL 2002 by Eugene Charniak and Dekang Lin, and
earlier ACL and EACL formats,  which were written by several people,
including John Chen, Henry S. Thompson and Donald Walker. Additional elements
were taken from the formatting instructions of the {\em International Joint
Conference on Artificial   Intelligence} and the \emph{Conference on Computer
Vision and Pattern Recognition}.

\bibliography{references}
\bibliographystyle{acl_natbib}
\end{document}


